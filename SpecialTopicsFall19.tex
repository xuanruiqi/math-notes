\documentclass[12pt, a4paper, titlepage]{report}

\usepackage{amsmath}
\usepackage{amssymb}
\usepackage{amsfonts}
\usepackage{xfrac}
\usepackage{amsthm}
\usepackage{dsfont}
\usepackage{thmtools}
\usepackage{nameref, hyperref, cleveref}
\usepackage{wasysym}

\declaretheorem[numberwithin=section, title=Theorem, refname={theorem, theorems},
Refname={Theorem, Theorems}]{thm}
\declaretheorem[sibling=thm, title=Lemma, refname={lemma, lemmas},
Refname={Lemma, Lemmas}]{lem}
\declaretheorem[sibling=thm, title=Proposition]{prop}
\declaretheorem[sibling=thm, title=Corollary]{cor}

\declaretheorem[numberwithin=section, title=Definition, style=definition]{defn}
\declaretheorem[sibling=defn, title=Conjecture, style=definition]{conj}
\declaretheorem[sibling=defn, title=Example, style=definition]{exmp}

\declaretheorem[numbered=no, title=Remark, style=remark]{rem}
\declaretheorem[numbered=no, title=Note, style=remark]{note}

% \theoremstyle{plain} % default
% \newtheorem{thm}{Theorem}[section]
% \newtheorem{lem}[thm]{Lemma}
% \newtheorem{prop}[thm]{Proposition}
% \newtheorem{cor}[thm]{Corollary}

% \theoremstyle{definition}
% \newtheorem{defn}{Definition}[section]
% \newtheorem{conj}{Conjecture}[section]
% \newtheorem{exmp}{Example}[section]

% \theoremstyle{remark}
% \newtheorem*{rem}{Remark}
% \newtheorem*{note}{Note}

\DeclareMathOperator{\Ker}{Ker}
\DeclareMathOperator{\Ima}{Im}
\DeclareMathOperator{\Endo}{End}
\DeclareMathOperator{\chara}{char}
\DeclareMathOperator{\vspan}{span}

\newcommand\opp[1]{{#1}^{\mathrm{op}}}
\newcommand\dangle[1]{\langle {#1} \rangle}


\title{Topics in Mathematical Science VI, Fall 2019: Module Theory and Homological Algebra}
\date{\today}
\author{Erik Darp\"{o}, Graduate School of Mathematics, Nagoya University
  \and Notes by: Xuanrui Qi}

\begin{document}

\maketitle
\tableofcontents

\chapter{Rings and modules}

\section{Review of basic ring theory}

To introduce the concept of modules, we must first introduce the concept of the ring, which
should be covered in any undergraduate algebra course. Here, we revisit the definition:

\begin{defn}[ring]
  A \textbf{ring} is a triple $(R, +, \cdot)$ where $R$ is a set and $+$ and $\cdot$ are operations on $R$, i.e.
  $R \times R \rightarrow R$, satisfying the following axioms:
  \begin{enumerate}
  \item $\exists 0 \in R$, $\forall a \in R$, $0 + a = a$ (existence of additive neutral element)
  \item $\forall a \in R$, $\exists b \in R$, $a + b = 0$ (additive inverse)
  \item $\forall a, b, c \in R$, $a + (b + c) = (a + b) + c$ (associativity of addition)
  \item $\forall a, b \in R$, $a + b = b + a$ (commutativity of addition). In other words, $(R, +)$ forms a
    abelian group.
  \item $\exists 1 \in R$, $\forall a \in R$, $1 \cdot a = a$ (existence of multiplicative neutral element)
  \item $\forall a, b, c \in R$, $a \cdot (b \cdot c) = (a \cdot b) \cdot c$ (associativity of multiplication). In other
    words, $(R, \cdot)$ forms a monoid.
  \item $\forall a, b, c \in R$, $a(b + c) = ab + ac$ and $(a + b)c = ac + bc$ (distributivity of multiplication over
    addition).
  \end{enumerate}
\end{defn}

\begin{rem}
  This is the definition of a ring with unity. Some authors do not require that the multiplicative identity exists,
  but hereafter whenever we refer to a ``ring'', we always mean a ring with unity.
\end{rem}

\begin{rem}
  In axiom 1, the additive neutral element $0$ is always unique. The proof is left as an exercise for the reader.
\end{rem}

\begin{rem}
  In axiom 2, $b$ is always uniquely determined by $a$. For this reason we usually denote $b$ as $-a$.
\end{rem}

\begin{rem}
  It could be easily shown that $\forall a \in R, 0 \cdot a = 0$, and that $\forall a \in R, (-1) \cdot a = -a$.
\end{rem}

\begin{defn}
  A ring is \textbf{commutative} if $\forall a, b \in R, ab = ba$.
\end{defn}

\begin{exmp}
  Here are some examples of rings:
  \begin{enumerate}
  \item $\mathbb{Z}$, $\mathbb{Q}$, $\mathbb{R}$, $\mathbb{C}$ and $\mathbb{Z}/n\mathbb{Z}$ form rings under usual
    addition and multiplication. These rings are all commutative;
  \item let $R$ be a ring and $n \in \mathbb{Z}^{+}$. Then, the set of $n \times n$ matrices, $R^{n \times n}$, forms a
    (non-commutative) ring under elementwise addition and matrix multiplication;
  \item let $R$ be a ring. We define a \textbf{polynomial} over $R$ as the symbolic expression
    \[
      \sum_{i = 0}^{n} a_{i}x^{i}
    \]
    where $a_i \in R$ and $n \in \mathbb{Z}^{+}$.
    The set of polynomials over $R$, or $R[x]$, forms a ring. Addition is defined as elementwise addition, and multiplication
    follows the usual rules (i.e., as in the multiplication of real polynomials);
  \item let $X$ be a set and $R$ a ring, then
    \[
      R^{X} = \{ f \mid f : X \rightarrow R \}
    \]
    is also a ring, where addition and multiplication are defined pointwise. $R^{X}$ and $R[x]$ are commutative if and only if
    $R$ is commutative.
  \end{enumerate}
\end{exmp}

\begin{defn}
  Let $R$ be a ring and $S \subseteq R$. $S$ is a \textbf{subring} of $R$ if:
  \begin{enumerate}
  \item $1_{R} \in S$
  \item $\forall a, b \in S$, $a + b \in S$
  \item $\forall a \in S$, $-a \in S$
  \item $\forall a, b \in S$, $ab \in S$
  \end{enumerate}
\end{defn}

\begin{defn}
  Let $R$ be a ring. The \textbf{group of units} in $R$ is the set
  $R^{\times} = \{ a \in R \mid \exists b \in R, ab = 1 \}$. It is easy to
  verify that it forms a group under multiplication in $R$.
\end{defn}

\begin{defn}
  A ring is a \textbf{division ring} in case that $R^{\times} = R \backslash \{0\}$.
  A \textbf{field} is a division ring that is commutative.
\end{defn}

\section{Modules over a ring}
Now that we have reviewed the basics of ring theory, we can give the definition of a module.

\begin{defn}[module]
  Let $R$ be a ring. A (left) $R$-\textbf{module} is a pair $(M, \cdot)$ where $M = (M, +)$ is an abelian group
  and $\cdot$ is an operation $R \times M \rightarrow M$, $(a, m) \mapsto am$, often called \textit{scalar multiplication},
  satisfying the following axioms:
  \begin{enumerate}
  \item $\forall a \in R$, $\forall m, n \in M$, $a(m +_{M} n) = am + an$
  \item $\forall a, b \in R$, $\forall m \in M$, $(a +_{R} b)m = am + bm$
  \item $\forall a, b \in R$, $\forall m \in M$, $(ab)m = a(bm)$
  \item $\forall m \in M$, $1_{R}m = m$
  \end{enumerate}
\end{defn}

\begin{rem}
  These axioms are exactly the same axioms as that of an vector space, except that in the definition of a vector space
  the ring $R$ is further limited to a field. 
\end{rem}

Alternatively, we can give a definition of modules in term of morphism groups:

\begin{rem}
  Let M be an $R$-module. Every $a \in R$ determines a map $\rho_a : M \rightarrow M$, $x \mapsto ax$. It is easy to verify that
  $\rho_a$ is a group morphism, if $(M, +)$ is viewed as an abelian group.

  Then, we can define a map $\rho : R \rightarrow \Endo(M, +)$, $a \mapsto \rho_a$, where $\Endo(M, +)$ is the set of
  group (endo)morphisms $(M, +) \rightarrow (M, +)$.

  Next, we shall verify that $\Endo(M, +)$ is a ring. First, we note that it is an abelian group under pointwise addition of
  morphisms.

  We then define multiplication as $\phi \cdot \psi = \phi \circ \psi$ (i.e., function composition). It is easy to see that under
  pointwise addition as addition, and composition as multiplication, $\Endo(M, +)$ is a ring, called the \textbf{endomorphism ring}
  of $(M, +)$. The proof is left as an exercise for the reader.

  Therefore, axiom 2 tells us that $\rho_{a + b}(m) = (a+b)m = am + bm = \rho_a(m) + \rho_b(m)$, i.e., $\rho_{a + b} = \rho_a + \rho_b$, i.e.
  $\rho$ is a group morphism. Furthermore, axioms 3 and 4 tell us that $\rho$ is a ring morphism; the proof is left as an exercise
  for the reader.

  As such, we can equivalently define a $R$-module $M$ to be an abelian group equipped with a ring morphism
  $\rho : R \rightarrow \Endo(M, +)$.
\end{rem}

\begin{exmp}
  Here are some examples of modules:
  \begin{enumerate}
  \item if $K$ is a field, then a $K$-module is exactly a $K$ vector space;
  \item a $\mathbb{Z}$-module is exactly an abelian group. The proof is omitted here but is not difficult, and
    can be an easy exercise;
  \item let $R$ be a ring. A natural example of a module would be the vectors $R^n$, which form an $R$-module;
  \item a $K[x]$-module is a $K$-vector space $V$ equipped with a linear map $V \rightarrow V$.
  \end{enumerate}
\end{exmp}

\begin{defn}
  Let $M$ be an $R$-module and $A$ a subgroup of $A$. We call $A$ a \textbf{submodule} of $M$ if
  $\forall r \in R$, $\forall x \in A$, $rx \in A$.
\end{defn}

The definition we have just given is for a \textbf{left} module. Dually, we can define a \textbf{right} module, where scalar
multiplication operates on the right:

\begin{defn}[right module]
  Let $R$ be a ring. A right $R$-module is a pair $(M, \cdot)$ where $M = (M, +)$ is an abelian group
  and $\cdot$ is an operation $M \times R \rightarrow M$, $(x, r) \mapsto xr$, satisfying the
  following axioms:
  \begin{enumerate}
  \item $\forall m, n \in M$, $\forall a \in R$, $(m + n)a = ma + na$
  \item $\forall m \in M$, $\forall a, b \in R$, $m(a + b) = ma + mb$
  \item $\forall m \in M$, $\forall a, b \in R$, $m(ab) = (ma)b$
  \item $\forall m \in M$, $m \cdot 1_{R} = m$
  \end{enumerate}
\end{defn}

However, every left module is equivalent to a right module, and vice versa. To show this, we need to introduce the concept of an
\textbf{opposite ring}:

\begin{defn}
  Let $R$ be a ring. The \textbf{opposite ring} of $R$, $\opp{R} = (R, \ast)$, where $\opp{R}$ is the same abelian group as $R$, but with the
  multiplication operation $\ast$ defined as $a \ast b = b_Ra_R$.
\end{defn}

Then, any right $R$-module is equivalently a left $\opp{R}$-module under the scalr multiplication $\opp{R} \times M \rightarrow M$,
$(r, x) \mapsto r * x = xr$.
Furthermore, if $R$ is a commutative ring, then $\opp{R} = R$, so in this case the left and right $R$-modules are exactly the same.

\section{Constructions on modules}

There are a number of basic constructions on $R$-modules that yield new $R$-modules.

\begin{thm}
  The intersection of an arbitrary number of submodules of $M$ is again a submodule of $M$.
\end{thm}

\begin{proof}
  Left as an exercise for the reader. Use the definition of a submodule.
\end{proof}

The next construction on submodules is the \textit{sum} construction. We give the definition as following:

\begin{defn}(sum of submodules)
  Let $M$ be an $R$-module and $\{A_i\}_{i \in I}$ be an $I$-indexed family of submodules of $M$. The \textbf{sum} of the
  modules $A_1$, $A_2$, ..., is the set
  \[
    \{ \sum_{i \in I} a_i = a_1 + a_2 + ... \mid a_i \in A_i \}.
  \]
\end{defn}

\begin{thm}
  Let $M$ be an $R$-module. The sum of a family of submodules of $M$ is again a submodule of $M$.
\end{thm}

\begin{proof}
  Left as an exercise for the reader. Use the definition of a submodule.
\end{proof}

Submodules can also be generated by subsets of a module, similar to how subsets of a linear space can span a linear subspace.

\begin{defn}
  Let $M$ be an $R$-module and $X \subset M$. Then the \textbf{submodule generated by $X$}, or $\dangle{X}$, is the set:
  \[
    \{ \sum_{i \in I} r_ix_i \mid r_i = 0 \text{ for almost all } i \in I \}
  \]
  where $I$ is an indexing set, and $r_i$ and $x_i$ include all elements in $R$ and $X$, respectively.
\end{defn}

It is easy to show that $\dangle{X}$ is indeed a submodule of $M$.

\begin{rem}
  Here, ``almost all'' means ``except for finitely many''. Apparently, the sum needs to be finite for this to make sense, hence
  the ``almost all'' condition. Furthermore, when $R$ and $X$ are finite, this condition vanishes.
\end{rem}

Consider an $R$-module $M$ and a submodule $N$. Consider $M$ and $N$ as abelian groups; we can form the quotient (or factor)
group $M/N$. From group theory, we know that the elements of $M/N$ are the equivalence classes $[x]$, where $x \in M$.
We claim that $M/N$ is also an $R$-module.

\begin{thm}
  Let $M$ be an $R$-module and $N \subset M$ a submodule of $M$. Then the quotient group $M/N$ is also an $R$ module, with
  scalar defined as: $\forall [x] \in M/N$, $\forall r \in R$, $r[x] = [rx]$. Therefore, we call $M/N$ the \textbf{quotient module}
  of $M$ over $N$.
\end{thm}

\begin{proof}
  It is easy to verify that this construction satisfies all the module axioms using the properties of quotient groups.
\end{proof}

Finally, we define three more constructions on modules, which are direct analogies from constructions on vector spaces in linear algebra:

\begin{defn}
  Let $(M_i)_{i \in I}$ be an indexed family of $R$-modules. The \textbf{direct product} of $(M_i)_{i \in I}$ is defined as:
  \[
    \prod_{i \in I} M_i = \{ (x_i)_{i \in I} \mid x_i \in M_i \}.
  \]
\end{defn}

It is trivial to verify that this is an $R$-module, with both addition and scalar multiplication defined elementwise.

\begin{exmp}
  If $I$ is a countable set (say $I = \{ 1, 2, ..., n \}$), then
  \[
    \prod_{i \in I} M_i = M_1 \times M_2 \times ... M_n
  \]
  is simply the Cartesian product. This is apparently an $R$-module, with the scalar product defined as
  \[
    r(x_1, x_2, ..., x_n) = (rx_1, rx_2, ..., rx_n).
  \]

  Therefore, the direct product is a generalization of the usual Cartesian product.
\end{exmp}

\begin{defn}
  Let $(M_i)_{i \in I}$ be an indexed family of $R$-modules. The \textbf{coproduct} of $(M_i)_{i \in I}$ is defined as:
  \[
    \coprod_{i \in I} M_i = \{ (x_i)_{i \in I} \in \prod_{i \in I} M_i \mid x_i \text{ for almost all } i \in I \}.
  \]
  That is, a coproduct is a product in which only finitely many elements are nonzero.
\end{defn}

\begin{rem}
  $\coprod_{i \in I} M_i \subseteq \prod_{i \in I}$ and is a submodule of the latter.
\end{rem}

\begin{rem}
  In the finite case, the direct product and the coproduct coincide. That is,
  $\coprod_{i \in I} M_i = \prod_{i \in I}$.
\end{rem}

The last construction we introduce is the \textit{direct sum}, which generates the relationship between
a vector space and its basis in linear algebra.

\begin{defn}
  Let $M$ be an $R$-module. $M$ is a \textbf{direct sum} of the submodules $L_1, ..., L_n \subseteq M$
  if $\forall m \in M$, $\exists! x_i \in L_i$ (i.e., exists one and only one $x_i \in L_i$),
  where $i \in \{ 1, ..., n \}$, such that
  \[
    m = x_1 + ... + x_n
  \]

  In this case, we write that:
  \[
    M = L_1 \oplus ... \oplus L_n
  \]
\end{defn}

\begin{rem}
  If $M$ is the direct sum of a family of submodules, then it is also the sum of this family of submodules.
\end{rem}

\begin{rem}
  If $M = L_1 \oplus L_2$, then $M = L_1 + L_2$ and $L_1 \cap L_2 = \{0\}$.
\end{rem}

\begin{rem}
  Let $M_1, ..., M_n$ be $R$-modules, then
  \[
    \widetilde{M}_i = 0 \times ... \times M_i \times 0 \times ... \times 0 \subseteq \prod_{i \in I} M_i
  \]
  is an $R$-module. Moreover,
  \[
    M_1 \times ... \times M_n = \widetilde{M}_1 \oplus ... \oplus \widetilde{M}_n.
  \]

\end{rem}

\section{Ideals}

\begin{defn}
  Let $R$ be a ring:

  \begin{enumerate}
  \item a \textbf{left ideal} in $R$ is a submodule of $R$ when viewed as a left module over itself, i.e.
    $I \subseteq {}_RR$;
  \item a \textbf{right ideal} in $R$ is a submodule of $R$ when viewed as a right module over itself, i.e.
    $I \subseteq R_R$;
  \item a \textbf{(two-sided) ideal} in $R$ is both a left ideal and a right ideal in $R$. In this case, we
    write $I \lhd R$.
  \end{enumerate}
\end{defn}

\begin{rem}

  If we plug in the definition of a submodule, then we obtain the more familiar definition of a submodule. If $R$ is a
  ring, then $I \subseteq R$ is a left ideal in $R$ if

  \begin{enumerate}
  \item $0 \in I$;
  \item $\forall x, y \in I$, $x + y \in I$;
  \item $\forall r \in R$, $\forall x \in I$, $rx \in I$.
  \end{enumerate}

  We could do the same to obtain the more familiar definition of a right ideal.
  
\end{rem}

\begin{exmp}
  Let $R$ be a ring and $R^{2 \times 2}$ the ring of all $2 \times 2$ matrices in $R$. The set

  \[
    \begin{pmatrix}
      R & 0\\
      R & 0
    \end{pmatrix} =
    \Big\{
    \begin{pmatrix}
      x & 0\\
      y & 0
    \end{pmatrix}
    \Big|
    r, s \in R
    \big\} \subset R^{2 \times 2}
  \]
  is a left ideal in $R$, but \textit{not} a right ideal in $R$, because:

  \[
    \begin{pmatrix}
      1 & 0\\
      1 & 0
    \end{pmatrix}
    \begin{pmatrix}
      0 & 1\\
      0 & 0
    \end{pmatrix} =
    \begin{pmatrix}
      0 & 1\\
      0 & 1
    \end{pmatrix} \notin
    \begin{pmatrix}
      R & 0\\
      R & 0
    \end{pmatrix}
  \]

  Similarly,
  \[
    \begin{pmatrix}
      R & R\\
      0 & 0
    \end{pmatrix}
  \]
  is a right, but not left, ideal in $R$.
  
\end{exmp}

\begin{defn}
  Let $R$ be a commutative ring and let $a \in R$ an element of $R$. Then
  \[
    (a) = Ra = \{ ra \mid r \in R \} \lhd R
  \]
  is called the \textbf{principal ideal} determined by $a$.
\end{defn}

\begin{rem}
  Apparently, this is a two-sided ideal in $R$. The proof is trivial.
\end{rem}

\begin{exmp}
  Let $K$ be a field and $a \in K$. Then
  \[
    \{ f(x) \in K[x] \mid f(a) = 0 \}
  \]
  is a principal ideal in $K[x]$, generated by $x - a$.
\end{exmp}

\begin{lem}
  Let $R$ be a ring and $I \lhd R$, then the quotient group $R/I$ is a left $R$-module, with
  scalar multiplication defined as
  \[
    r \cdot [x] = [rx]
  \]
  as well as a right $R$-module, with scalar multiplication defined as
  \[
    [x] \cdot r = [xr]
  \]
  as well as a ring, with multiplication defined as
  \[
    [x][y] = [xy].
  \]
\end{lem}

\begin{rem}
  Therefore, we also call $R/I$ a \textbf{quotient ring}.
\end{rem}

\begin{lem}
  A proper left or right ideal $I \subset R$ cannot contain an invertible element. In other words, if
  an ideal $I \subseteq R$ contains an invertible element, then $I = R$.
\end{lem}

\begin{proof}
  Suppose $x \in I$ is invertible, and let $a$ be any element in $R$, then by definition $(ax^{-1})x \in I$.
  However, $(ax^{-1})x = a(x^{-1}x) = a$, so $\forall a \in R$, $a \in I$, i.e., $I = R$.
\end{proof}

\begin{cor}
  Any division ring $D$ has only two ideals, the zero ideal $\{0\}$ and the trivial ideal which is $D$ itself.
\end{cor}

\begin{rem}
  The ideals in $\mathbb{Z}$ are exactly the subsets $n\mathbb{Z} = \{ na \mid a \in \mathbb{Z} \}$.
\end{rem}

\section{Module morphisms}

Like any other algebraic structure, modules come with morphisms between them, defined in a natural manner.

\begin{defn}
  Let $R$ be a ring and $M, N \in \mathrm{Mod} \ R$. A map $f : M \rightarrow N$ is a \textbf{morphism of $R$-modules},
  or an \textbf{$R$-linear map}, if $\forall x, y \in M$, $r \in R$,

  \begin{enumerate}
  \item $f(x + y) = f(x) + f(y)$, and
  \item $f(rx) = r \cdot f(x)$.
  \end{enumerate}
\end{defn}

Now, we shall introduce some terminology related to module morphisms.

\begin{defn}
  Let $f : M \rightarrow N$ be a morphism of $R$-modules. We say that $f$ is:
  \begin{enumerate}
  \item a \textbf{monomorphism}, in case that it is injective. Here,
    we also write that $f : M \rightarrowtail N$;
  \item an \textbf{epimorphism}, in case that it is surjective. Here,
    we also write that $f : M \twoheadrightarrow N$;
  \item an \textbf{isomorphism}, in case that it is bijective. Here,
    we also write that $f : M \xrightarrow{\sim} N$;
  \item an \textbf{endomorphism}, in case that $M = N$;
  \item an \textbf{automorphism}, in case that it is endo and iso.
  \end{enumerate}
\end{defn}

\begin{rem}
  Let $L \subset M$ be an inclusion of $R$-modules. Consider the \textit{inclusion map} from
  $L$ to $M$:
  \[
    \iota : L \rightarrow M, \iota = x \mapsto x
  \]
  Here, $\iota$ is clearly mono.

  Meanwhile, the map
  \[
    \pi : M \rightarrow M/L, \pi = x \mapsto [x]
  \]
  is clearly an epi.
\end{rem}

\begin{rem}
  Every morphism $f : M \rightarrow N$ of $R$-modules determines:

  \begin{enumerate}
  \item a submodule $\Ker f \subseteq M$;
  \item a submodule $\Ker f \subseteq N$.
  \end{enumerate}
\end{rem}

The next result is similar to well-known, analogous theorems about ring, field and linear space morphisms:

\begin{lem}
  \label{lem:morph}
  Let $f : M \rightarrow N$ be a morphism of $R$-modules, then:

  \begin{enumerate}
  \item $f$ is mono if and only if $\Ker f = \{0\}$;
  \item $f$ is an epi if and only if $\Ima f = N$;
  \item $f$ is iso if and only if $\exists g : M \rightarrow N$ a $R$-module morphism,
    such that $g \circ f = \mathds{1}_M$ and $f \circ g = \mathds{1}_N$. In other words,
    a morphism is iso in case that it is invertible.
  \end{enumerate}
\end{lem}

\section{The isomorphism theorems}
There are three isomorphism theorems on modules, analogous to the isomorphism theorems for groups and/or rings.
Let $M$ and $N$ be $R$-modules; in this section, we will denote the inclusion map $M \hookrightarrow N$
as $\iota$, and the coset map $M \to M/N$, $x \mapsto [x]$, as $\pi$, without further explanation. 

\begin{thm}[the first isomorphism theorem for modules]
  Every morphism $f : M \to N$ of $R$-modules determines an isomorphism $\overline{f} : M/\Ker f \to \Ima f$,
  such that $f = \iota \circ \overline{f} \circ \pi$.
\end{thm}

\begin{proof}
  We note that $\overline{f} = [x] \mapsto f(x)$, where $x \in M$. Thus, we need to show that:
  \begin{enumerate}
  \item $\overline{f}$ is well-defined (i.e., indeed a map);
  \item $\overline{f}$ is a morphism;
  \item $\overline{f}$ is bijective (i.e., iso);
  \item that $f = \iota \circ \overline{f} \circ \pi$.
  \end{enumerate}

  We prove these goals one by one.

  \begin{enumerate}
  \item
    For $\overline{f}$ to be well defined, we need to show that, if $[x] = [y]$ in $M/\Ker f$, then
    $f(x) = f(y)$.

    If $[x] = [y]$, then $x + \Ker f = y + \Ker f$, which means that $x - y \in \Ker f$, by the definition of
    a coset. Moreover, as $f(x-y) = f(x) - f(y) = 0$, we know that $f(x) = f(y)$.

  \item It is trivial to show that $\overline{f}$ is a morphism of $R$-modules. The interested reader may
    verify it using definitions.

  \item
    We would like to show that $\overline{f}$ is both injective and surjective. First we shall prove the injectivity
    of $\overline{f}$ using \autoref{lem:morph}.

    Suppose $\overline{f}([x]) = 0$, then we have $f(x) = 0$, that is $x \in \Ker f$. In other words, $[x] = [0]$.
    Therefore, $\Ker \overline{f} = \{ [0] \}$, so by \autoref{lem:morph} $\overline{f}$ is injective.

    The surjectivity of $\overline{f}$ is easy to prove. By definition, for every $y \in \Ima f$, $\exists x \in M$
    such that $f(x) = y$. That is, there is a coset $[x]$ such that $\overline{f}([x]) = y$.

  \item
    Having proven the first three items, this is easy to verify. For each $x \in M$,
    \[
      \iota \circ \overline{f} \circ \pi(x) = \iota(\overline{f}(\pi(x))) = \iota(\overline{f}([x])) = \iota(f(x)) = f(x)
    \]
    thus $\iota \circ \overline{f} \circ \pi = f$.
  \end{enumerate}
\end{proof}

The next two isomorphism theorems will be given without proof.

\begin{thm}[the second isomorphism theorem for modules]
  Let $N$ be an $R$-module and $L, M \subseteq N$ submodules of $N$. Then the map
  \begin{align*}
    L \big/ (L \cap M) &\to (L + M) \big/ M\\
    [x] &\mapsto [x]
  \end{align*}
  is an isomorphism.
\end{thm}

\begin{thm}[the third isomorphism theorem for modules]
  Let $L \subseteq M \subseteq N$ be an inclusion of $R$-modules. Then $M/L$ is a submodule of $N/L$,
  and furthermore the map
  \begin{align*}
    (N/L) \big/ (M/L) &\to N \big/ M\\
    (x + L) + (M/L) &\mapsto x + M
  \end{align*}
  is an isomorphism.
\end{thm}

Now, we shall revisit the familiar concept of a ring homomorphism.

\begin{defn}[homomorphism of rings]
  Let $R, S$ be rings. A ring homomorphism is a map $f : R \to S$, such that $\forall a, b \in R$:
  \begin{enumerate}
  \item $f(a+b) = f(a) + f(b)$;
  \item $f(ab) = f(a)f(b)$;
  \item $f(1_R) = 1_S$.
  \end{enumerate}
\end{defn}

\begin{rem}
  For every ring morphism $f : R \to S$, $\Ker f \lhd R$ is a (two-sided) ideal, and $\Ima f \subseteq S$ is a
  subring.
\end{rem}

Since every ring is a module over itself, all three isomorphism theorems apply to rings too, with ``submodule''
replaced by ``ideal''. The proof is of course trivial, so we will not repeat them here.

\begin{exmp}
  For any ring $R$, there exists a unique isomorphism
  \[
    \chi : \mathbb{Z} \to R, \chi = n \mapsto n1_R = \overbrace{1_R + ... + 1_R}^{n \text{ times}}.
  \]
  
  Since $\Ker \chi \lhd \mathbb{Z}$, so $\Ker \chi = m\mathbb{Z}$ for some $m \in \mathbb{N}$.
  Moreover, by the first isomorphism theorem we know that $\chi$ determines an isomorphism
  $\overline{\chi} : \mathbb{Z}/m\mathbb{Z} \to \mathbb{Z}1_R$. This (uniquely determined) $m \in \mathbb{N}$ is
  called the \textbf{characteristic} of $R$, or $\chara R$.
\end{exmp}

\begin{cor}
  The characteristic of a field is either 0 or a prime number.
\end{cor}

\begin{proof} Left as an exercise for the reader. \end{proof}

\section{Finitely generated modules and finite dimensional algebras}

\begin{defn}
  Let $M$ be an $R$-module. We say that $M$ is \textbf{finitely generated} if $\exists x_1, ..., x_n \in M$, such
  that $M = \dangle{\{x_1, ..., x_n\}}$. Often, we will omit the curly braces and just write that
  $M = \dangle{x_1, ..., x_n}$.

  We denote the collection (or category) of finitely generated $R$-modules as $\text{mod } R$.
\end{defn}

\begin{exmp}
  Let $R = K$ be a field, then a $R$-module is a vector space $V$ over $K$. $V$ is finitely generated if and only if
  $\exists x_1, ..., x_n \in V$ such that $V = \vspan_K \{x_1, ..., x_n\}$. From linear algebra, we know that we can always
  reduce a spanning set to a basis, so $V$ must have a finite basis $u = (u_1, ..., u_l)$ where $l \leq n$.

  Therefore, a $K$-vector space is a finite generated $K$-module if and only if it is a finite dimensional vector space.
\end{exmp}

\begin{exmp}
  Let $R$ be a commutative ring, and let $\Lambda = R[x]$. Then

  \begin{enumerate}
  \item $\dangle{1_\Lambda} = \Lambda$ is a finitely generated $\Lambda$-module;
  \item let $a \in R$, $M = R$, and define scalar multiplication as
    \begin{align*}
      \Lambda \times M &\to M\\
      (f(x), M) &\mapsto f(x) \cdot m := f(a) \cdot m.
    \end{align*}

    Then, $M = R$ is a $\Lambda$-module.
  \end{enumerate}
\end{exmp}

\begin{prop}
  Let $M$ be an $R$-module. $M$ is finitely generated if and only if there exists an $n \in \mathbb{Z}_{+}$, such that
  there exists an epimorphism ${}_R R \times ... \times {}_R R \twoheadrightarrow M$.
\end{prop}

\begin{defn}
  If $M = \dangle{x_1, ..., x_n}$, we define the map $e_i : {}_R R \times ... \times {}_R R \to M$ to be
  \[
    e_i = (0, ..., 1, ..., 0) \mapsto x_i
  \]
  where on the left-hand side the $i$-th element is 1.
\end{defn}

\begin{defn}
  Let $K$ be a field. A $K$-algebra is a ring $\Lambda$ with a map $K \times \Lambda \to \Lambda$, $(c, \lambda) \mapsto c\lambda$
  making $\Lambda$ a $K$-vector space, such that $\forall x, y \in \Lambda$, $\forall c \in K$, $c(xy) = (cx)y = x(cy)$.

  In particular, $\cdot$ in $\Lambda$ is bilinear. That is, $\forall a \in \Lambda$,
  \[
    L_a : \Lambda \to \Lambda, L_a = x \mapsto ax
  \]
  and
  \[
    R_a : \Lambda \to \Lambda, R_a = x \mapsto xa
  \]
  are both $K$-linear maps.

  We define the dimension of the $K$-algebra $\Lambda$, or $\dim_K \Lambda$, as the dimension of $\Lambda$ as a vector space. We say that
  $\Lambda$ is a finite-dimensional $K$-algebra if it is a finite-dimensional vector space.
\end{defn}

\begin{rem}
  Any $K$-algebra is a ring, and the map $K \to \Lambda$, $a \mapsto a \cdot 1_{\Lambda}$ is a ring homomorphism.
\end{rem}

\begin{rem}
  Let $R$ be a ring, $K$ a field, and $\varepsilon : K \to R$ a ring homomorphism. Then $R$ becomes a $K$-algebra, with $\cdot$ defined as
  \begin{align*}
    K \times R &\to R\\
    (c, r) &\mapsto \varepsilon(c) \cdot r = cr.
  \end{align*}
\end{rem}

\begin{rem}
  Let $\Lambda$ be a $K$-algebra and $M$ a $\Lambda$-module, then $M$ is also a $K$-module, via
  \begin{align*}
    K \times M &\to M\\
    (c, m) &\mapsto (c \cdot 1_R) \cdot m = cm.
  \end{align*}
\end{rem}

\begin{rem}
  If $\Lambda$ is a finite dimensional $K$-algebra, then
  \[
    \text{mod } \Lambda = \{ M \in \text{Mod } \Lambda \mid \dim_K M < \infty \}
  \]
  i.e., the finitely generated $\Lambda$-modules are exactly the modules that are also finite dimensional $K$-algebras.
\end{rem}

\section{Chain conditions}
In this section we define the \textit{chain conditions} on modules, concerning chains of module inclusions.

\begin{defn}
  An $R$-module $M$ is \textbf{Noetherian} if the \textbf{ascending chain condition}, explained below, holds.

  Let $M_0 \subseteq M_1 \subseteq M_2 \subseteq ...$ be a chain of inclusion of submodules of $M$, i.e., $M_0$ is a submodule of $M_1$,
  $M_1$ a submodule of $M_2$, and so on. We say that $R$ satisfies the ascending chain condition if for any such chain,
  $\exists N \in \mathbb{N}$, $\forall n \geq N$, we have $M_n = M_N$. That is, all ascending chains in $M$ terminate.

  Noetherian modules are named after Emmy Noether, who first studied the ascending chain condition in detail.
\end{defn}

Dually, there are also descending chains and a similar condition.

\begin{defn}
  An $R$-module $M$ is \textbf{Artinian}, named after Emil Artin, if the \textbf{descending chain condition} holds.

  Let $M_0 \supseteq M_1 \supseteq M_2 \supseteq ...$ be a chain of reverse inclusion of submodules of $M$, i.e., $M_1$ is a submodule of $M_0$,
  $M_2$ a submodule of $M_1$, and so on. We say that $R$ satisfies the descending chain condition if for any such chain,
  $\exists N \in \mathbb{N}$, $\forall n \geq N$, we have $M_n = M_N$. That is, all descending chains in $M$ terminate.
\end{defn}

Since rings are modules over themselves, we can also define the chain condition for rings.

\begin{defn}
  A ring is left (or right) Noetherian/Artinian if ${}_R R$ (or $R_R$) is Noetherian/Artinian.
\end{defn}

\begin{defn}
  A ring is Noetherian/Artinian if it is both left and right Noetherian/Artinian.
\end{defn}

\begin{rem}
  A finite module is apparently both Noetherian and Artinian.
\end{rem}

\begin{prop}
  If $\Lambda$ is a finite dimensional $K$-algbera, then every finitely-generated $\Lambda$-module is both Noetherian and Artinian.
\end{prop}

\begin{prop}
  $\mathbb{Z}$ is a Noetherian ring.
\end{prop}

\begin{proof}
  Let us write $(n) = n\mathbb{Z}$, and let $(n_0) \subseteq (n_1) \subseteq (n_2) \subseteq ...$ be an ascending chain in $\mathbb{Z}$.

  Let $i < j \in \mathbb{N}$, and we have $(n_i) \subseteq (n_j) = \{ mn_j \mid m \in \mathbb{Z} \}$. Therefore, $n_i = mn_j$ for some $m \in \mathbb{Z}$,
  i.e. $n_j \mid n_i$.

  Since $n_j < n_i$, we know that $\exists N \in \mathbb{N}$, such that $\forall j \geq N$, $n_j = n_N$ (because every integer has a unique and finite
  prime factorization).
\end{proof}

\begin{rem}
  However, $\mathbb{Z}$ is not Artinian. For example, the chain
  \[
    \mathbb{Z}/2\mathbb{Z} \supseteq \mathbb{Z}/4\mathbb{Z} \supseteq \mathbb{Z}/8\mathbb{Z} \supseteq ...
  \]
  does not terminate.
\end{rem}

\begin{exmp}
  Let $M = \mathbb{Q}/\mathbb{Z}$ be a $\mathbb{Z}$-module, and $p$ a prime number. Then
  \[
    M_p = \big\{ \big[ \frac{m}{p^a} \big] \in M \mid m \in \mathbb{Z}, a \in \mathbb{N} \big\}
  \]
  is Artinian but not Noetherian.
\end{exmp}

\begin{proof} Left as an exercise for the reader. \end{proof}

\begin{lem}
  Let $R$ be a ring and $M$ an $R$-module, then the following are equivalent:

  \begin{enumerate}
  \item $M$ is Noetherian;
  \item every submodule of $M$ is finitely generated;
  \item for every submodule $L \subseteq M$, $L$ and $M/L$ are both Noetherian.
  \end{enumerate}
\end{lem}

\end{document}