\documentclass[12pt, a4paper]{article}

\usepackage{amsmath}
\usepackage{amssymb}
\usepackage{amsfonts}
\usepackage{dsfont}

\usepackage{wasysym}



\newcommand{\Endo}[1]{\ensuremath{\mathrm{End}(#1)}}
\newcommand\opp[1]{{#1}^{\mathrm{op}}}
\newcommand\dangle[1]{\langle {#1} \rangle}


\title{An alternative proof of convergence for $e^x$}
\date{\today}
\author{Xuanrui Qi}

\begin{document}
\maketitle

Define the exponential function $f(x) = e^x$ as
\[
  e^x = \sum_{n = 0}^{\infty} \frac{x^n}{n!}.
\]

We would like to show that $e^x$ is indeed defined on all of $\mathbb{R}$, i.e.,
$e^x$ converges for every $x \in \mathbb{R}$. Normally, we would use the ratio test to
show that $e^x$ is indeed convergent:
\[
  \lim_{n \to \infty} \lvert \frac{a_{n+1}}{a_n} \rvert = \lim_{n \to \infty} \lvert \frac{\frac{x^{n+1}}{(n+1)!}}{\frac{x^n}{n!}} \rvert
  = \lim_{n \to \infty} \lvert \frac{x}{n+1} \rvert = 0 < 1
\]
therefore $e^x$ converges (and in fact, converges absolutely) for all $x \in \mathbb{R}$.

However, this proof requires setting up the theory of sequences, series, convergence and absolute convergence, and then we can finally
prove the basic convergence tests and thus the convergence of $e^x$ (essentially, most of Chapter 3 of Rudin's \textit{Principles of Mathematical Analysis}),
which is much too complicated if we want to define and introduce the exponential function as soon as possible.

There is a simple but quite clever proof that requires nothing other than the definition of convergence of an infinite series,
the comparison test (which is trivial to prove), and some elementary results. This proof was first introduced to me by Serge Richard.

First, we note a simple fact: for every $x \in \mathbb{R}$, there exists a $N \in \mathbb{N}$ such that $x/N < 1$. One could prove
this using the axioms of real numbers if one would like to get very rigorous, but we will take this for granted here.
Then, we have:

\[
  e^x = \sum_{n = 0}^{N} \frac{x^n}{n!} + \sum_{n = N+1}^{\infty} \frac{x^n}{n!}
\]

The first part of the sum is apparently finite, and we can safely ignore it. Thus, our task is to prove that
\[
  S = \sum_{n = N+1}^{\infty} \frac{x^n}{n!}
\]
converges.

The key observation here is that
\[
  n! = n(n-1)(n-2)...(N+1)N(N-1)...1 = n(n-1)...(N+1)N!,
\]
so
\[
  \frac{x^n}{n!} = \frac{x}{n} \frac{x}{n-1} ... \frac{x}{N+1} \frac{x^N}{N!} < \frac{x}{N} \frac{x}{N} ... \frac{x}{N} \frac{x^N}{N!}
\]

It seems that we need a few more $N$'s in the denominator. But we can do this:
\[
  \frac{x^n}{n!} = N^{N}\frac{x^n}{N^{N}n!} =  N^{N}\frac{x}{n} \frac{x}{n-1} ... \frac{x}{N+1} \frac{x}{N} ... \frac{x}{N} \frac{1}{N!} <  N^{N} \frac{x}{N} \frac{x}{N} ... \frac{x}{N} ...
  = N^N (\frac{x}{N})^n.
\]
To show that the inequality holds, we only need two observations: that for every $n > N$, $x/n < x/N$, and that $1/N! < 1$. Both are trivial.

From the previous inequality, we can deduce that
\[
  S = \sum_{n = N+1}^{\infty} \frac{x^n}{n!} < \sum_{n = N+1}^{\infty} N^N (\frac{x}{N})^n = N^N \sum_{n = N+1}^{\infty} (\frac{x}{N})^n = S'
\]
using the so-called comparison test.

Now, note that the right hand side is simply a geometric series (multiplied by a constant)! We know that the geometric series
\[
  \sum_{n = k}^{\infty} a^n
\]
converges whenever $a < 1$.

Since we know that $x/N < 1$, $S'$ is convergent. Moreover, as $S < S'$, $S$ also converges. As a result, we know that
\[
  \sum_{n = 0}^{N} \frac{x^n}{n!} + S
\]
converges for all $x$. That is, for each and every $x \in \mathbb{R}$, $e^x$ converges.

\end{document}